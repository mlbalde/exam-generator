%@Specs(BeforeHead=1)
%@Header.begin
%% Based on a TeXnicCenter-Template by Gyorgy SZEIDL.
%%%%%%%%%%%%%%%%%%%%%%%%%%%%%%%%%%%%%%%%%%%%%%%%%%%%%%%%%%%%%

%------------------------------------------------------------
%
\documentclass{article}%
%Options -- Point size:  10pt (default), 11pt, 12pt
%        -- Paper size:  letterpaper (default), a4paper, a5paper, b5paper
%                        legalpaper, executivepaper
%        -- Orientation  (portrait is the default)
%                        landscape
%        -- Print size:  oneside (default), twoside
%        -- Quality      final(default), draft
%        -- Title page   notitlepage, titlepage(default)
%        -- Columns      onecolumn(default), twocolumn
%        -- Equation numbering (equation numbers on the right is the default)
%                        leqno
%        -- Displayed equations (centered is the default)
%                        fleqn (equations start at the same distance from the right side)
%        -- Open bibliography style (closed is the default)
%                        openbib
% For instance the command
%           \documentclass[a4paper,12pt,leqno]{article}
% ensures that the paper size is a4, the fonts are typeset at the size 12p
% and the equation numbers are on the left side
%
\usepackage{amsmath}%
\usepackage{amsfonts}%
\usepackage{amssymb}%
%\usepackage{graphicx}
\usepackage{listings}
\usepackage{alltt}
%-------------------------------------------
%\newtheorem{theorem}{Theorem}
%\newtheorem{acknowledgement}[theorem]{Acknowledgement}
%\newtheorem{algorithm}[theorem]{Algorithm}
%\newtheorem{axiom}[theorem]{Axiom}
%\newtheorem{case}[theorem]{Case}
%\newtheorem{claim}[theorem]{Claim}
%\newtheorem{conclusion}[theorem]{Conclusion}
%\newtheorem{condition}[theorem]{Condition}
%\newtheorem{conjecture}[theorem]{Conjecture}
%\newtheorem{corollary}[theorem]{Corollary}
%\newtheorem{criterion}[theorem]{Criterion}
%\newtheorem{definition}[theorem]{Definition}
%\newtheorem{example}[theorem]{Example}
%\newtheorem{exercise}[theorem]{Exercise}
%\newtheorem{lemma}[theorem]{Lemma}
%\newtheorem{notation}[theorem]{Notation}
%\newtheorem{problem}[theorem]{Problem}
%\newtheorem{proposition}[theorem]{Proposition}
%\newtheorem{remark}[theorem]{Remark}
%\newtheorem{solution}[theorem]{Solution}
%\newtheorem{summary}[theorem]{Summary}
%\newenvironment{proof}[1][Proof]{\textbf{#1.} }{\ \rule{0.5em}{0.5em}}

\lstset{basicstyle=\small,numbers=left,language=Java,aboveskip=0.5em,belowskip=0.5em,frame=single,
	xleftmargin=10pt,framexleftmargin=2em, showstringspaces=false,breaklines=true,
	escapeinside={(*@}{@*)} }
\begin{document}

Test subjects\\

Some text\\
Please take care of the following points
\begin{enumerate}
\item Point 1,
with addons.
\item Point 2,
\item Point 3 with the following subpoints:
	\begin{enumerate}
	\item Subpoint 1, %@Specs(InsideHead=1)
	\item Subpoint 2,
	\item Subpoint 3.
\end{enumerate}
\end{enumerate}

Subject no: @Ref

%@Header.end
\begin{enumerate}
%@Specs(AfterHead=1)
%@QuestionBlock.begin(questions=auto)

Given the code
\begin{lstlisting}
%\newtheorem{theorem}{Theorem}
%\newtheorem{acknowledgement}[theorem]{Acknowledgement}
%\newtheorem{algorithm}[theorem]{Algorithm}
%\newtheorem{axiom}[theorem]{Axiom}
%\newtheorem{case}[theorem]{Case}
%\newtheorem{claim}[theorem]{Claim}
%\newtheorem{conclusion}[theorem]{Conclusion}
%\newtheorem{condition}[theorem]{Condition}
%\newtheorem{conjecture}[theorem]{Conjecture}
\end{lstlisting}

\item Say which affirmations are correct
\begin{enumerate}
\item \verb+v1+ Answer 1
\item \verb+v1+ Answer 2
\item \verb+v1+ Answer 3
\item \verb+v1: %@Begin.header + Answer 4 %@T because I want it to be
\item \verb+v1+ Answer 5
\item \verb+v1+ Answer 6
\item \verb+v1+ Answer 7
\item \verb+v1+ Answer 8		%@T Just to give an example
\item \verb+v1+ Answer 9
\item \verb+v1+ Answer 10
\end{enumerate}
\item Say which affirmations are correct
\begin{enumerate}	

\item \verb+v1+ Answer 1
\item \verb+v1+ Answer 2
\item \verb+v1+ Answer 3
\item \verb+v1: %@Header.end + Answer 4 %@T because I want it to be
\item \verb+v1+ Answer 5
\item \verb+v1+ Answer 6
\item \verb+v1+ Answer 7
\item \verb+v1+ Answer 8		%@T Just to give an example
\item \verb+v1+ Answer 9
\item \verb+v1+ Answer 10	
\end{enumerate}
%@QuestionBlock.end
%@QuestionBlock.begin()
Let it be the code:
\begin{lstlisting}
	\begin{theorem}
	(The Currant minimax principle.) Let $T$ be completely continuous selfadjoint operator
	in a Hilbert space $H$. Let $n$ be an arbitrary integer and let $u_1,\ldots,u_{n-1}$ be
	an arbitrary system of $n-1$ linearly independent elements of $H$. Denote
	\begin{equation}
	\max_{\substack{v\in H, v\neq
	0\\(v,u_1)=0,\ldots,(v,u_n)=0}}\frac{(Tv,v)}{(v,v)}=m(u_1,\ldots, u_{n-1})
	\label{eqn10}
	\end{equation}
	Then the $n$-th eigenvalue of $T$ is equal to the minimum of these maxima, when
	minimizing over all linearly independent systems $u_1,\ldots u_{n-1}$ in $H$,
	\begin{equation}
	\mu_n = \min_{\substack{u_1,\ldots, u_{n-1}\in H}} m(u_1,\ldots, u_{n-1}) \label{eqn20}
	\end{equation}
	\end{theorem}
	The above equations are automatically numbered as equation (\ref{eqn10}) and
	(\ref{eqn20}).

\end{lstlisting}
\item Question 1\\
\begin{enumerate}
	\item \verb+v1+ Answer 1
	\item \verb+v1+ Answer 2
	\item \verb+v1+ Answer 3
	\item \verb+v1: %@Header.end + Answer 4 %@T because I want it to be
	\item \verb+v1+ Answer 5
	\item \verb+v1+ Answer 6
	\item \verb+v1+ Answer 7
	\item \verb+v1+ Answer 8		%@T Just to give an example
	\item \verb+v1+ Answer 9
	\item \verb+v1+ Answer 10	
\end{enumerate}

\item Question 2
\begin{enumerate}
	\item \verb+v1+ Answer 1
	\item \verb+v1+ Answer 2
	\item \verb+v1+ Answer 3
	\item \verb+v1: %@Header.end + Answer 4 %@T because I want it to be
	\item \verb+v1+ Answer 5
	\item \verb+v1+ Answer 6
	\item \verb+v1+ Answer 7
	\item \verb+v1+ Answer 8		%@T Just to give an example
	\item \verb+v1+ Answer 9
	\item \verb+v1+ Answer 10	
\end{enumerate}
%@QuestionBlock.end

\item Question 3
\begin{enumerate}
	\item \verb+v1+ Answer 1
	\item \verb+v1+ Answer 2
	\item \verb+v1+ Answer 3
	\item \verb+v1: %@Header.end + Answer 4 %@T because I want it to be
	\item \verb+v1+ Answer 5
\end{enumerate}

\item Question 4:
\begin{enumerate}
	\item \verb+v1+ Answer 1
	\item \verb+v1+ Answer 2
	\item \verb+v1+ Answer 3
	\item \verb+v1: %@Header.end + Answer 4 %@T because I want it to be
	\item \verb+v1+ Answer 5
\end{enumerate}

\item Question 5: No correct answers
\begin{enumerate}
	\item \verb+v1+ Answer 1
	\item \verb+v1+ Answer 2
	\item \verb+v1+ Answer 3
	\item \verb+v1:+ Answer 4 
	\item \verb+v1+ Answer 5
\end{enumerate}



\end{enumerate}

%\title{This is the Title of a Standard \LaTeX\ Article}
%\author{J. A. Smith\thanks{This is for making an acknowledgement.}
%\\The University of Miskolc, Hungary}
%\date{February 24, 2002}
%\maketitle
%
%\begin{abstract}
%This is a sample document which shows the most important features of the Standard
%\LaTeX\ Journal Article class.
%\end{abstract}
%
%\section{Introduction}
%
%\noindent The front matter has various entries such as\\
%\hspace*{\fill}\verb" \title", \verb"\author", \verb"\date", and
%\verb"\thanks"\hspace*{\fill}\\
%You should replace their arguments with your own.
%
%This text is the body of your article. You may delete everything between the commands\\
%\hspace*{\fill} \verb"\begin{document}" \ldots \verb"\end{documant}"
%\hspace*{\fill}\\in this file to start with a blank document.
%
%
%\section{The Most Important Features}
%
%\noindent Sectioning commands. The first one is the\\
%\hspace*{\fill} \verb"\section{The Most Important Features}" \hspace*{\fill}\\
%command. Below you shall find examples for further sectioning commands:
%
%\subsection{Subsection}
%Subsection text.
%
%\subsubsection{Subsubsection}
%Subsubsection text.
%
%\paragraph{Paragraph}
%Paragraph text.
%
%\subparagraph{Subparagraph}Subparagraph text.\vspace{2mm}
%
%Select a part of the text then click on the button Emphasize (H!), or Bold (Fs), or
%Italic (Kt), or Slanted (Kt) to typeset \emph{Emphasize}, \textbf{Bold},
%\textit{Italics}, \textsl{Slanted} texts.
%
%You can also typeset \textrm{Roman}, \textsf{Sans Serif}, \textsc{Small Caps}, and
%\texttt{Typewriter} texts.
%
%You can also apply the special, mathematics only commands $\mathbb{BLACKBOARD}$
%$\mathbb{BOLD}$, $\mathcal{CALLIGRAPHIC}$, and $\mathfrak{fraktur}$. Note that
%blackboard bold and calligraphic are correct only when applied to uppercase letters A
%through Z.
%
%You can apply the size tags -- Format menu, Font size submenu -- {\tiny tiny},
%{\scriptsize scriptsize}, {\footnotesize footnotesize}, {\small small}, {\normalsize
%normalsize}, {\large large}, {\Large Large}, {\LARGE LARGE}, {\huge huge} and {\Huge
%Huge}.
%
%You can use the \verb"\begin{quote} etc. \end{quote}" environment for typesetting
%short quotations. Select the text then click on Insert, Quotations, Short Quotations:
%
%\begin{quote}
%The buck stops here. \emph{Harry Truman}
%
%Ask not what your country can do for you; ask what you can do for your
%country. \emph{John F Kennedy}
%
%I am not a crook. \emph{Richard Nixon}
%
%I did not have sexual relations with that woman, Miss Lewinsky. \emph{Bill Clinton}
%\end{quote}
%
%The Quotation environment is used for quotations of more than one paragraph. Following
%is the beginning of \emph{The Jungle Books} by Rudyard Kipling. (You should select
%the text first then click on Insert, Quotations, Quotation):
%
%\begin{quotation}
%It was seven o'clock of a very warm evening in the Seeonee Hills when Father Wolf woke
%up from his day's rest, scratched himself, yawned  and spread out his paws one after
%the other to get rid of sleepy feeling in their tips. Mother Wolf lay with her big gray
%nose dropped across her four tumbling, squealing cubs, and the moon shone into the
%mouth of the cave where they all lived. ``\emph{Augrh}'' said Father Wolf, ``it is time
%to hunt again.'' And he was going to spring down hill when a little shadow with a bushy
%tail crossed the threshold and whined: ``Good luck go with you, O Chief of the Wolves;
%and good luck and strong white teeth go with the noble children, that they may never
%forget the hungry in this world.''
%
%It was the jackal---Tabaqui the Dish-licker---and the wolves of India despise Tabaqui
%because he runs about making mischief, and telling tales, and eating rags and pieces of
%leather from the village rubbish-heaps. But they are afraid of him too, because
%Tabaqui, more than any one else in the jungle, is apt to go mad, and then he forgets
%that he was afraid of anyone, and runs through the forest biting everything in his way.
%\end{quotation}
%
%Use the Verbatim environment if you want \LaTeX\ to preserve spacing, perhaps when
%including a fragment from a program such as:
%\begin{verbatim}
%#include <iostream>         // < > is used for standard libraries.
%void main(void)             // ''main'' method always called first.
%{
% cout << ''This is a message.'';
%                            // Send to output stream.
%}
%\end{verbatim}
%(After selecting the text click on Insert, Code Environments, Code.)
%
%
%\subsection{Mathematics and Text}
%
%It holds \cite{KarelRektorys} the following
%\begin{theorem}
%(The Currant minimax principle.) Let $T$ be completely continuous selfadjoint operator
%in a Hilbert space $H$. Let $n$ be an arbitrary integer and let $u_1,\ldots,u_{n-1}$ be
%an arbitrary system of $n-1$ linearly independent elements of $H$. Denote
%\begin{equation}
%\max_{\substack{v\in H, v\neq
%0\\(v,u_1)=0,\ldots,(v,u_n)=0}}\frac{(Tv,v)}{(v,v)}=m(u_1,\ldots, u_{n-1})
%\label{eqn10}
%\end{equation}
%Then the $n$-th eigenvalue of $T$ is equal to the minimum of these maxima, when
%minimizing over all linearly independent systems $u_1,\ldots u_{n-1}$ in $H$,
%\begin{equation}
%\mu_n = \min_{\substack{u_1,\ldots, u_{n-1}\in H}} m(u_1,\ldots, u_{n-1}) \label{eqn20}
%\end{equation}
%\end{theorem}
%The above equations are automatically numbered as equation (\ref{eqn10}) and
%(\ref{eqn20}).
%
%\subsection{List Environments}
%
%You can create numbered, bulleted, and description lists using the tag popup
%at the bottom left of the screen.
%
%\begin{enumerate}
%\item List item 1
%
%\item List item 2
%
%\begin{enumerate}
%\item A list item under a list item.
%
%The typeset style for this level is different than the screen style. \ The
%screen shows a lower case alphabetic character followed by a period while the
%typeset style uses a lower case alphabetic character surrounded by parentheses.
%
%\item Just another list item under a list item.
%
%\begin{enumerate}
%\item Third level list item under a list item.
%
%\begin{enumerate}
%\item Fourth and final level of list items allowed.
%\end{enumerate}
%\end{enumerate}
%\end{enumerate}
%\end{enumerate}
%
%\begin{itemize}
%\item Bullet item 1
%
%\item Bullet item 2
%
%\begin{itemize}
%\item Second level bullet item.
%
%\begin{itemize}
%\item Third level bullet item.
%
%\begin{itemize}
%\item Fourth (and final) level bullet item.
%\end{itemize}
%\end{itemize}
%\end{itemize}
%\end{itemize}
%
%\begin{description}
%\item[Description List] Each description list item has a term followed by the
%description of that term. Double click the term box to enter the term, or to
%change it.
%
%\item[Bunyip] Mythical beast of Australian Aboriginal legends.
%\end{description}
%
%\subsection{Theorem-like Environments}
%
%The following theorem-like environments (in alphabetical order) are available
%in this style.
%
%\begin{acknowledgement}
%This is an acknowledgement
%\end{acknowledgement}
%
%\begin{algorithm}
%This is an algorithm
%\end{algorithm}
%
%\begin{axiom}
%This is an axiom
%\end{axiom}
%
%\begin{case}
%This is a case
%\end{case}
%
%\begin{claim}
%This is a claim
%\end{claim}
%
%\begin{conclusion}
%This is a conclusion
%\end{conclusion}
%
%\begin{condition}
%This is a condition
%\end{condition}
%
%\begin{conjecture}
%This is a conjecture
%\end{conjecture}
%
%\begin{corollary}
%This is a corollary
%\end{corollary}
%
%\begin{criterion}
%This is a criterion
%\end{criterion}
%
%\begin{definition}
%This is a definition
%\end{definition}
%
%\begin{example}
%This is an example
%\end{example}
%
%\begin{exercise}
%This is an exercise
%\end{exercise}
%
%\begin{lemma}
%This is a lemma
%\end{lemma}
%
%\begin{proof}
%This is the proof of the lemma.
%\end{proof}
%
%\begin{notation}
%This is notation
%\end{notation}
%
%\begin{problem}
%This is a problem
%\end{problem}
%
%\begin{proposition}
%This is a proposition
%\end{proposition}
%
%\begin{remark}
%This is a remark
%\end{remark}
%
%\begin{solution}
%This is a solution
%\end{solution}
%
%\begin{summary}
%This is a summary
%\end{summary}
%
%\begin{theorem}
%This is a theorem
%\end{theorem}
%
%\begin{proof}
%[Proof of the Main Theorem]This is the proof.
%\end{proof}
%\medskip
%
%This text is a sample for a short bibliography. You can cite a book by making use of
%the command \verb"\cite{KarelRektorys}": \cite{KarelRektorys}. Papers can be cited
%similarly: \cite{Bertoti97}. If you want multiple citations to appear in a single set
%of square brackets you must type all of the citation keys inside a single citation,
%separating each with a comma. Here is an example: \cite{Bertoti97, Szeidl2001,
%Carlson67}.
%
%\begin{thebibliography}{9}                                                                                                %
%\bibitem {KarelRektorys}Rektorys, K., \textit{Variational methods in Mathematics,
%Science and Engineering}, D. Reidel Publishing Company,
%Dordrecht-Hollanf/Boston-U.S.A., 2th edition, 1975
%
%\bibitem {Bertoti97} \textsc{Bert\'{o}ti, E.}:\ \textit{On mixed variational formulation
%of linear elasticity using nonsymmetric stresses and displacements}, International
%Journal for Numerical Methods in Engineering., \textbf{42}, (1997), 561-578.
%
%\bibitem {Szeidl2001} \textsc{Szeidl, G.}:\ \textit{Boundary integral equations for
%plane problems in terms of stress functions of order one}, Journal of Computational and
%Applied Mechanics, \textbf{2}(2), (2001), 237-261.
%
%\bibitem {Carlson67}  \textsc{Carlson D. E.}:\ \textit{On G\"{u}nther's stress functions
%for couple stresses}, Quart. Appl. Math., \textbf{25}, (1967), 139-146.
%\end{thebibliography}
%
%
%\appendix
%
%\section{The First Appendix}
%
%The appendix fragment is used only once. Subsequent appendices can be created
%using the Section Section/Body Tag.
%@Footer.begin

\end{document}
%@Footer.end