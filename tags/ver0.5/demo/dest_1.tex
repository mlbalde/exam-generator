%
\documentclass[a4paper,10pt]{article}%

\usepackage{amsmath}%
\usepackage{amsfonts}%
\usepackage{amssymb}%
\usepackage{graphicx}
\usepackage{listings}
\usepackage{alltt}
\usepackage{fullpage}
\usepackage{multicol}
\usepackage{tikz}
\usepackage{forloop}
\usepackage{supertabular}
\usepackage{epstopdf}
\usepackage{enumitem}
\usepackage{multirow}
\usepackage{array}
\usepackage{balance}

\usetikzlibrary{shapes.geometric}
\usetikzlibrary{patterns}

\usepackage[left=1.5cm,top=1.5cm,right=1.5cm,bottom=2cm,footskip=14pt, nohead]{geometry}

\renewcommand{\figurename}{Figura }

\lstset{basicstyle=\small,numbers=left,language=Java,aboveskip=0.5em,belowskip=0.5em,frame=single,
	xleftmargin=10pt,framexleftmargin=2em, showstringspaces=false,breaklines=true,
	escapeinside={(*@}{@*)},
	 tabsize=2
	 }

\begin{document}

\begin{center}
\Large 
Demo question sheet presenting main facilities
\end{center}

\normalsize
This text will be replaced: Replaced!. The searched token @\hspace{0 pt}Rep. Please note that this token will be replaced even if it is in verbatim environment (\verb+\verb!...!+).  This is because the replacement is done from java code, executing a simple header wide text replace. \\ 

This is a "source" document containing several sections, questions and question block that demonstrate several capabilities of the quiz generator.

This section is beween \verb+%@Header.begin+ and \verb+%@Header.end+ tags. \\

An example of answer grid. A student might check the correct choices here.

\vspace{0.2 cm}
\begin{center}
\Large
\begin{multicols}{2}
	\begin{supertabular}{|c|c|c|c|c|c|}
		\hline	&A&B&C&D&E \\ \hline
	 	\newcounter{qid}
	 	\forloop{qid}{1}{\value{qid} < 10 }{\arabic{qid} &  &  &  & & \\  \hline}
	 	\arabic{qid} &  &  &  & &\\ \hline
	\end{supertabular}

	\columnbreak
	
	\begin{supertabular}{|c|c|c|c|c|c|}
		\hline	&A&B&C&D&E \\ \hline
	 	\forloop{qid}{11}{\value{qid} < 20 }{\arabic{qid} &  &  &  & & \\  \hline}
	 	\arabic{qid} &  &  &  & &\\ \hline
	\end{supertabular}
	
\end{multicols}
\end{center}

\normalsize

\vspace{1cm}
One can write here whatever instructions are needed. 

This document can be used as a skeleton to create your own "question database".

\large
Another reference that can be replaced: Replaced!. This part is still in the @Header.
\normalsize

In the header is a good place to write some driving parameters. Number of questions in the generated output, number of choices per question, maximum correct choices, etc.

Here, the following @Specs tag is placed:\\ \verb+%@Specs(TotalCount=auto AnswersPerQestion=5 MaxTrue=5 MinTrue=1 WriteAnswers=0)+. Note that in the source code we have two @Specs, one that is for ExamGenerator to interpret and one is for printing. The latter is written in a verbatim envoironment so the software will skip it. \\



Please note that usually the number of questions in a quiz is fixed.

 \setlength{\itemsep}{1pt}
 \setlength{\parskip}{0pt}
 \setlength{\parsep}{0pt}

%Here, we specify the end of header
\section{First section}
It is tagged with "DoNotRandomize=1" so it will be always the first section in the output. \\The tag is: \verb+%@Params (DoNotRandomize=1, questions=6)+ and is placed before the enumerate environment.

Another annotation is the number of questions that will be generated. This section can generate 1+3+4=8 questions but we want only 6.

\begin{enumerate}
	
\item  This is question 2 with 15 answers. In the output this question will be broken and shuffled with other questions in this section
	
\begin{enumerate}
\item  Choice 3, marked as true. %
		\item  Choice 5, marked as true. %
		\item  Choice 11,
		\item  Choice 8,
		\item  Choice 12,
		\end{enumerate}
\item  This is question 3 with 20 answers. In the output this question will be broken and shuffled with other questions in this section
	
\begin{enumerate}
\item  Choice 20, %		
	\item  Choice 15
		\item  Choice 14
		\item  Choice 1,
		\item  Choice 11,
		\end{enumerate}
\item  This is question 1 with 5 answers.
	
\begin{enumerate}
\item  Choice 5,
	\item  Choice 3, 
		\item  Choice 2, marked as true. %
		\item  Choice 4,
		\item  Choice 1, marked as true. %
		\end{enumerate}
\item  This is question 3 with 20 answers. In the output this question will be broken and shuffled with other questions in this section
	
\begin{enumerate}
\item  Choice 2,
		\item  Choice 6,
		\item  Choice 8,
		\item  Choice 17,
		\item  Choice 16,
		\end{enumerate}
\item  This is question 3 with 20 answers. In the output this question will be broken and shuffled with other questions in this section
	
\begin{enumerate}
\item  Choice 5, marked as true. %
		\item  Choice 13
		\item  Choice 9,
		\item  Choice 10,
		\item  Choice 12,
		\end{enumerate}
\item  This is question 2 with 15 answers. In the output this question will be broken and shuffled with other questions in this section
	
\begin{enumerate}
\item  Choice 9,
		\item  Choice 6,
		\item  Choice 7,
		\item  Choice 15
	\item  Choice 2,
		\end{enumerate}
\end{enumerate}
\section{Second section}
This section has no DoNotRanomize so its position will be shuffled.

Please note the usage of [resume].
\begin{enumerate}[resume]
	
\item  This is question 1 section 2, with 5 answers.
	
\begin{enumerate}
\item  Choice 5,
	\item  Choice 2, marked as true. %
		\item  Choice 1, marked as true. %
		\item  Choice 3, 
		\item  Choice 4,
		\end{enumerate}
\end{enumerate}
\section{Fifth section}
This section has no DoNotRanomize so its position will be shuffled.

We also exemplify a question block. Please note that the questions in the questionblock will not be shuffled.
\begin{enumerate}[resume]
	
	
\item  This is question 2 with 20 answers. In the output this question will be broken and shuffled with other questions in this section\\
	Placed here just to have enough material to shuffle with.
	
\begin{enumerate}
\item  Choice 16,
		\item  Choice 5, marked as true. %
		\item  Choice 14
		\item  Choice 13
		\item  Choice 20, %		
	\end{enumerate}
%
		This is a question block. Text placed here will be produced in the output. Mainly we have a problem and several questions related to it.
		
		
	

\item  Question inside the block, nr 2.
		
\begin{enumerate}
\item  Choice 4, marked as true. %
			\item  Choice 3, marked as true. %
			\item  Choice 2,
			\item  Choice 5, marked as true. %
		\item  Choice 1,
			\end{enumerate}
\item  Large question inside the block, nr 1.
		
\begin{enumerate}
\item  Choice 2,
			\item  Choice 3, marked as true. %
			\item  Choice 1,
			\item  Choice 7,
			\item  Choice 6,
			\end{enumerate}
\item  Large question inside the block, nr 1.
		
\begin{enumerate}
\item  Choice 10,
		\item  Choice 9,
			\item  Choice 8,
			\item  Choice 5, marked as true. %
			\item  Choice 4, marked as true. %
			\end{enumerate}
\item  This is question 1 section 2, with 5 answers.
	
\begin{enumerate}
\item  Choice 3, 
		\item  Choice 5,
	\item  Choice 1, marked as true. %
		\item  Choice 2, marked as true. %
		\item  Choice 4,
		\end{enumerate}
\item  This is question 2 with 20 answers. In the output this question will be broken and shuffled with other questions in this section\\
	Placed here just to have enough material to shuffle with.
	
\begin{enumerate}
\item  Choice 15
		\item  Choice 3, marked as true. %
		\item  Choice 12,
		\item  Choice 1,
		\item  Choice 11,
		\end{enumerate}
\item  This is question 2 with 20 answers. In the output this question will be broken and shuffled with other questions in this section\\
	Placed here just to have enough material to shuffle with.
	
\begin{enumerate}
\item  Choice 6,
		\item  Choice 8,
		\item  Choice 18,
		\item  Choice 19,		
		\item  Choice 17,
		\end{enumerate}
\item  This is question 2 with 20 answers. In the output this question will be broken and shuffled with other questions in this section\\
	Placed here just to have enough material to shuffle with.
	
\begin{enumerate}
\item  Choice 10,
		\item  Choice 7,
		\item  Choice 2,
		\item  Choice 9,
		\item  Choice 4, marked as true. %
		\end{enumerate}
\end{enumerate}
\section{Fourth section}
This section has no DoNotRanomize so its position will be shuffled.
\begin{enumerate}[resume]
	
\item  This is question 1 section 2, with 5 answers.
	
\begin{enumerate}
\item  Choice 1, marked as true. %
		\item  Choice 5,
	\item  Choice 2, marked as true. %
		\item  Choice 4,
		\item  Choice 3, 
		\end{enumerate}
\end{enumerate}
\section{Third section}
This section has no DoNotRanomize so its position will be shuffled.
\begin{enumerate}[resume]
	
\item  This is question 1 section 3, with 5 answers.
	
\begin{enumerate}
\item  Choice 1, marked as true. %
		\item  Choice 4,
		\item  Choice 2, marked as true. %
		\item  Choice 3, 
		\item  Choice 5,
	\end{enumerate}
\end{enumerate}
\section{Last section}
This section has no DoNotRanomize so its position will be shuffled.
It is tagged with "DoNotRandomize=All,AlwaysSelect=1" so it will be always the last section in the output, and it will be always selected!

\begin{enumerate}[resume]
	
\item  Did you liked the exam? \\You don't want the order of the answers to be shuffled, so the section has DoNotRandomize=all parameter. If in this section were more questions, one might enclose only this question into a questionblock tagged appropriately.
	
\begin{enumerate}
\item  Hated it! %
		\item  Not quite
		\item  Tolerable
		\item  Liked it a bit.
		\item  Yes!
	\end{enumerate}
\end{enumerate}
%
Begin of the footer.
\end{document}
