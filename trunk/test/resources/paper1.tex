%regular comment %%%
%%%%%%%%%%%%%%%%%%%%  This text will be ignored.
%
%@Header.begin  The begining of the header.
\documentclass{article}
\usepackage{amsmath}
\usepackage{amsfonts}
\usepackage{amssymb}
\usepackage{graphicx}
\begin{document}
Some sample document

\verb!%@Header.begin! And some text after
\verb%  A real token! % %It's workinsg!
\lstinline!%@Header.begin! And some text after. The lstinline is the verb for listings package
\lstinline%  A real token! % %It's workinsg!

\begin{lstlisting}
package exam;
public class VarDeclare {
    public  static final Integer v1 = 20;
            static       String  s1 = "Ala Bala";%Comment
                         String  s2; 				%@Begin.Header
    private static       Integer v2 = 20;
    private              Integer v3 = 15;
    private        final Integer v4 = 23;
    public void Meth1(){
        v1=30;
        v2=20;
        v3=1;
    }
    private static Meth2(){
        v3=30;
        v4=30;
    }
}
\end{lstlisting}

\begin{verbatim}
#include <iostream>         // < > is used for standard libraries.
void main(void)             // ''main'' method always called first.
{
 cout << ''This is a message.'';
                            // Send to output stream.
}
\end{verbatim}
Please pay attention to the following:
\begin{enumerate}
	\item Ennumerate instances in the header should be parsed in verbatim
	\item Ennumerate instances in the header should be parsed in verbatim
	\item Ennumerate instances in the header should be parsed in verbatim
\end{enumerate}

%@Header.end

%@Specs(Attrib1=val1 Attrib2=val2 Attrib3=23) Some specifications. The comments after will be ignored.

\begin{enumerate}
\item Question 1
\begin{enumerate}
\item Answer 1
    \item Answer 2 %@T Because is true
    \item Answer 3 %Some other comments
\end{enumerate}
\item Question 2 %comment % asf
\begin{verbatim} 
asdfasdfasdfa
asdfasdfasd
% Some comment that is not actually comment
asd
\end{verbatim}

\begin{enumerate}
    \item Answer 1
    \item Answer 2
    \item Answer 3
    \item Answer 4
\end{enumerate}

%@QuestionBlock.begin()
	Quest Block. This text will be kept
	\item block Question 1
	\begin{enumerate}
		\item Answer 1
		\item Answer 2
		\item Answer 3
		\item Answer 4
	\end{enumerate}
	\item block Question 2
	\begin{enumerate}
		\item Answer 1
		\item Answer 2
		\item Answer 3
		\item Answer 4
	\end{enumerate}
%@QuestionBlock.end

%@QuestionBlock.begin(questions=10)
	Quest Block. This text will be kept
	\item block Question 3
	\begin{enumerate}
		\item Answer 1
		\item Answer 2
		\item Answer 3
		\item Answer 4
	\end{enumerate}
	\item block Question 4
	\begin{enumerate}
		\item Answer 1
		\item Answer 2
		\item Answer 3
		\item Answer 4
	\end{enumerate}
%@QuestionBlock.end


\end{enumerate}
%@Footer.begin
The same with the footer:
\begin{enumerate}
	\item Ennumerate instances in the header should be parsed in verbatim
	\item Ennumerate instances in the header should be parsed in verbatim
	\item Ennumerate instances in the header should be parsed in verbatim
\end{enumerate}
\end{document}
%@Footer.end

%Some text that will be ignored