%@Header.begin
\documentclass[a4paper,10pt]{article}%

\usepackage{amsmath}%
\usepackage{amsfonts}%
\usepackage{amssymb}%
\usepackage{graphicx}
\usepackage{listings}
\usepackage{alltt}
\usepackage{fullpage}
\usepackage{multicol}
\usepackage{tikz}
\usepackage{forloop}
\usepackage{supertabular}
\usepackage{epstopdf}
\usepackage{enumitem}
\usepackage{multirow}
\usepackage{array}
\usepackage{balance}

\usetikzlibrary{shapes.geometric}
\usetikzlibrary{patterns}

\usepackage[left=1.5cm,top=1.5cm,right=1.5cm,bottom=2cm,footskip=14pt, nohead]{geometry}

\renewcommand{\figurename}{Figura }

\lstset{basicstyle=\small,numbers=left,language=Java,aboveskip=0.5em,belowskip=0.5em,frame=single,
	xleftmargin=10pt,framexleftmargin=2em, showstringspaces=false,breaklines=true,
	escapeinside={(*@}{@*)},
	 tabsize=2
	 }

\begin{document}

\begin{center}
\Large 
Demo question sheet presenting main facilities
\end{center}

\normalsize
This text will be replaced: @Rep. The searched token @\hspace{0 pt}Rep. Please note that this token will be replaced even if it is in verbatim environment (\verb+\verb!...!+).  This is because the replacement is done from java code, executing a simple header wide text replace. \\ 

This is a "source" document containing several sections, questions and question block that demonstrate several capabilities of the quiz generator.

This section is beween \verb+%@Header.begin+ and \verb+%@Header.end+ tags. \\

An example of answer grid. A student might check the correct choices here.

\vspace{0.2 cm}
\begin{center}
\Large
\begin{multicols}{2}
	\begin{supertabular}{|c|c|c|c|c|c|}
		\hline	&A&B&C&D&E \\ \hline
	 	\newcounter{qid}
	 	\forloop{qid}{1}{\value{qid} < 10 }{\arabic{qid} &  &  &  & & \\  \hline}
	 	\arabic{qid} &  &  &  & &\\ \hline
	\end{supertabular}

	\columnbreak
	
	\begin{supertabular}{|c|c|c|c|c|c|}
		\hline	&A&B&C&D&E \\ \hline
	 	\forloop{qid}{11}{\value{qid} < 20 }{\arabic{qid} &  &  &  & & \\  \hline}
	 	\arabic{qid} &  &  &  & &\\ \hline
	\end{supertabular}
	
\end{multicols}
\end{center}

\normalsize

\vspace{1cm}
One can write here whatever instructions are needed. 

This document can be used as a skeleton to create your own "question database".

\large
Another reference that can be replaced: @Rep. This part is still in the @Header.
\normalsize

In the header is a good place to write some driving parameters. Number of questions in the generated output, number of choices per question, maximum correct choices, etc.

Here, the following @Specs tag is placed:\\ \verb+%@Specs(TotalCount=auto AnswersPerQestion=5 MaxTrue=5 MinTrue=1 WriteAnswers=0)+. Note that in the source code we have two @Specs, one that is for ExamGenerator to interpret and one is for printing. The latter is written in a verbatim envoironment so the software will skip it. \\

%@Specs(TotalCount=auto AnswersPerQestion=5 MaxTrue=5 MinTrue=0 WriteAnswers=0)

Please note that usually the number of questions in a quiz is fixed.

 \setlength{\itemsep}{1pt}
 \setlength{\parskip}{0pt}
 \setlength{\parsep}{0pt}

%Here, we specify the end of header
%@Header.end

All this text will be ignored when parsing the source. As a result it will not appear in the output file. Please note where in the source code this text is placed!


\section{First section}
It is tagged with "DoNotRandomize=1" so it will be always the first section in the output. \\The tag is: \verb+%@Params (DoNotRandomize=1, questions=6)+ and is placed before the enumerate environment.

Another annotation is the number of questions that will be generated. This section can generate 1+3+4=8 questions but we want only 6.
%@Params(DoNotRandomize=1  questions=6)
\begin{enumerate}
	\item This is question 1 with 5 answers.
	\begin{enumerate}
		\item Choice 1, marked as true. %@T
		\item Choice 2, marked as true. %@T
		\item Choice 3, 
		\item Choice 4,
		\item Choice 5,
	\end{enumerate}
	\item This is question 2 with 15 answers. In the output this question will be broken and shuffled with other questions in this section
	\begin{enumerate}
		\item Choice 1,
		\item Choice 2,
		\item Choice 3, marked as true. %@T
		\item Choice 4, marked as true. %@T
		\item Choice 5, marked as true. %@T
		\item Choice 6,
		\item Choice 7,
		\item Choice 8,
		\item Choice 9,
		\item Choice 10,
		\item Choice 11,
		\item Choice 12,
		\item Choice 13
		\item Choice 14
		\item Choice 15
	\end{enumerate}
	\item This is question 3 with 20 answers. In the output this question will be broken and shuffled with other questions in this section
	\begin{enumerate}
		\item Choice 1,
		\item Choice 2,
		\item Choice 3, marked as true. %@T
		\item Choice 4, marked as true. %@T
		\item Choice 5, marked as true. %@T
		\item Choice 6,
		\item Choice 7,
		\item Choice 8,
		\item Choice 9,
		\item Choice 10,
		\item Choice 11,
		\item Choice 12,
		\item Choice 13
		\item Choice 14
		\item Choice 15
		\item Choice 16,
		\item Choice 17,
		\item Choice 18,
		\item Choice 19,		
		\item Choice 20, %@T		
	\end{enumerate}
	
\end{enumerate}


\section{Second section}
This section has no DoNotRanomize so its position will be shuffled.

Please note the usage of [resume].
\begin{enumerate}[resume]
	\item This is question 1 section 2, with 5 answers.
	\begin{enumerate}
		\item Choice 1, marked as true. %@T
		\item Choice 2, marked as true. %@T
		\item Choice 3, 
		\item Choice 4,
		\item Choice 5,
	\end{enumerate}
\end{enumerate}



\section{Third section}
This section has no DoNotRanomize so its position will be shuffled.
\begin{enumerate}[resume]
	\item This is question 1 section 3, with 5 answers.
	\begin{enumerate}
		\item Choice 1, marked as true. %@T
		\item Choice 2, marked as true. %@T
		\item Choice 3, 
		\item Choice 4,
		\item Choice 5,
	\end{enumerate}
\end{enumerate}



\section{Fourth section}
This section has no DoNotRanomize so its position will be shuffled.
\begin{enumerate}[resume]
	\item This is question 1 section 2, with 5 answers.
	\begin{enumerate}
		\item Choice 1, marked as true. %@T
		\item Choice 2, marked as true. %@T
		\item Choice 3, 
		\item Choice 4,
		\item Choice 5,
	\end{enumerate}
\end{enumerate}



\section{Fifth section}
This section has no DoNotRanomize so its position will be shuffled.

We also exemplify a question block. Please note that the questions in the questionblock will not be shuffled.
\begin{enumerate}[resume]
	
	\item This is question 1 section 2, with 5 answers.
	\begin{enumerate}
		\item Choice 1, marked as true. %@T
		\item Choice 2, marked as true. %@T
		\item Choice 3, 
		\item Choice 4,
		\item Choice 5,
	\end{enumerate}
		Text that will be ignored, between questions\\
		
	\item This is question 2 with 20 answers. In the output this question will be broken and shuffled with other questions in this section\\
	Placed here just to have enough material to shuffle with.
	\begin{enumerate}	
		\item Choice 1,
		\item Choice 2,
		\item Choice 3, marked as true. %@T
		\item Choice 4, marked as true. %@T
		\item Choice 5, marked as true. %@T
		\item Choice 6,
		\item Choice 7,
		\item Choice 8,
		\item Choice 9,
		\item Choice 10,
		\item Choice 11,
		\item Choice 12,
		\item Choice 13
		\item Choice 14
		\item Choice 15
		\item Choice 16,
		\item Choice 17,
		\item Choice 18,
		\item Choice 19,		
		\item Choice 20, %@T		
	\end{enumerate}
	
	Text that will be ignored, before question block\\

	
	%@QuestionBlock.begin()
		This is a question block. Text placed here will be produced in the output. Mainly we have a problem and several questions related to it.
		
		\item Large question inside the block, nr 1.
		\begin{enumerate}
			\item Choice 1,
			\item Choice 2,
			\item Choice 3, marked as true. %@T
			\item Choice 4, marked as true. %@T
			\item Choice 5, marked as true. %@T
			\item Choice 6,
			\item Choice 7,
			\item Choice 8,
			\item Choice 9,
			\item Choice 10,
		\end{enumerate}
		\item Question inside the block, nr 2.
		\begin{enumerate}
			\item Choice 1,
			\item Choice 2,
			\item Choice 3, marked as true. %@T
			\item Choice 4, marked as true. %@T
			\item Choice 5, marked as true. %@T
		\end{enumerate}
	End of the question block. This text will also be ignored.
	%@QuestionBlock.end
	
\end{enumerate}



\section{Last section}
This section has no DoNotRanomize so its position will be shuffled.
It is tagged with "DoNotRandomize=All,AlwaysSelect=1" so it will be always the last section in the output, and it will be always selected!
%@Params(DoNotRandomize=All,AlwaysSelect=1)
\begin{enumerate}[resume]
	\item Did you liked the exam? \\You don't want the order of the answers to be shuffled, so the section has DoNotRandomize=all parameter. If in this section were more questions, one might enclose only this question into a questionblock tagged appropriately.
	\begin{enumerate}
		\item Hated it! %@T
		\item Not quite
		\item Tolerable
		\item Liked it a bit.
		\item Yes!
	\end{enumerate}
\end{enumerate}

This text will be also ignored. Look into the source file to see that is placed between the end of enumerate and beginning of the footer.

%@Footer.begin
Begin of the footer.
\end{document}
%@Footer.end